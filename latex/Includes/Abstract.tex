\begin{center}
	{\Large{\textbf{Prosody generation for Brazilian Portuguese in text-to-speech systems}}}
\end{center}

\vspace{1cm}

\begin{flushright}
	Author: Felipe Cortez de Sá \\
	Advisor: Carlos Augusto Prolo, Ph.D.
\end{flushright}

\vspace{1cm}

\begin{center}
	\Large{\textsc{\textbf{Abstract}}}
\end{center}

\noindent With the evergrowing presence of smartphones and home assistants in
our daily lives, technology companies have been developing two-way conversation
systems, that is, voice user interfaces. Despite its recent improvements,
text-to-speech programs still sound artificial, especially for their Brazilian
Portuguese voices. A big contributing factor for that is the lack of accurate
prosody, that is, pitch, length and emphasis. This thesis explores existing
text-to-speech systems, especially those for which there are Brazilian
Portuguese voices, focusing on their prosody generation modules. We highlight
challenges of prosody generation, review prosodic analysis in the Linguistics
field and propose possible solutions for improving text-to-speech quality.

% Com a cada vez mais forte presença de smartphones e home assistants no cotidiano, grandes empresas de tecnologia vêm desenvolvendo sistemas de conversação baseados em fala, denominadas voice user interfaces. Apesar dos avanços, é perceptível que os sistemas de síntese de voz, especialmente para o português brasileiro, deixam a desejar quanto à naturalidade da fala gerada. Um dos fatores principais que contribuem para isso é a prosódia, isto é, entonação, ritmo e acento da fala. Este trabalho investiga sistemas text-to-speech existentes através do estudo de seus algoritmos para síntese de voz e geração de prosódia para diversas línguas, com foco no português brasileiro. São explicitados os desafios encontrados, é feito um levantamento de modelos de análise prosódica na linguística e propõem-se possíveis soluções para tornar a geração de voz mais próxima à humana.

\noindent\textit{Keywords}: text-to-speech, prosody, voice user interfaces
