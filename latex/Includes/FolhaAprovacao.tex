% Folha de aprovação
\begin{folhadeaprovacao}
	\setlength{\ABNTsignthickness}{0.4pt}
	\setlength{\ABNTsignwidth}{10cm}
	
	% Informações gerais acerca do trabalho 
	% (nome do autor, título, instituição à qual é submetido e natureza)
	\noindent 
	Monografia de Graduação sob o título \textit{Um gerador de prosódia para o português brasileiro} apresentada por 
	Felipe Cortez de Sá e aceita pelo Departamento de Informática e Matemática Aplicada do
	Centro de Ciências Exatas e da Terra da Universidade Federal do Rio Grande do Norte,
	sendo aprovada por todos os membros da banca examinadora abaixo especificada:
		
	% Membros da banca examinadora e respectivas filiações
	\assinatura
	{
		Dr. Carlos Augusto Prolo\\
		{\small Orientador(a)} \\ 
		{\footnotesize
            Departamento de Informática e Matemática Aplicada\\
		  	Universidade Federal do Rio Grande do Norte
		}
	}
	
	% \assinatura
	% {
	% 	Titulação e nome do membro da banca examinadora	\\
	% 	{\small Co-orientador(a), se houver}\\ 
	% 	{\footnotesize
	% 		Departamento\\
	% 	  	Universidade
	% 	}
	% }
		
	\assinatura
	{
        Dr. Antônio Carlos Gay Thomé\\ 
		{\footnotesize
            Departamento de Informática e Matemática Aplicada\\
		  	Universidade Federal do Rio Grande do Norte
		}
	}
		
	\assinatura
	{
		The Third Man\\ 
		{\footnotesize
			Departamento\\
		  	Universidade
		}
	}
		
	\vfill
	
	\begin{center}
		Natal-RN, data de aprovação (por extenso).
	\end{center}
\end{folhadeaprovacao}
