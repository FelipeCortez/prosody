% Resumo em língua vernácula
\begin{center}
	{\Large{\textbf{\tccTitle}}}
\end{center}

\vspace{1cm}

\begin{flushright}
	Autor: Felipe Cortez de Sá \\
	Orientador(a): Dr. Carlos Augusto Prolo
\end{flushright}

\vspace{1cm}

\begin{center}
	\Large{\textsc{\textbf{Resumo}}}
\end{center}

\noindent Com a cada vez mais forte presença de smartphones e home assistants no
cotidiano, grandes empresas de tecnologia vêm desenvolvendo sistemas de
conversação baseados em fala, denominadas voice user interfaces. Apesar dos avanços, é perceptível que os sistemas de síntese de voz, especialmente para o português brasileiro, deixam a desejar quanto à naturalidade da fala gerada. Um dos fatores principais que contribuem para isso é a prosódia, isto é, entonação, ritmo e acento da fala. Este trabalho investiga sistemas text-to-speech existentes através do estudo de seus algoritmos para síntese de voz e geração de prosódia para diversas línguas, com foco no português brasileiro. São explicitados os desafios encontrados, é feito um levantamento de modelos de análise prosódica na linguística e propõem-se possíveis soluções para tornar a geração de voz mais próxima à humana.

\noindent\textit{Palavras-chave}: text-to-speech, prosódia, voice user interfaces
