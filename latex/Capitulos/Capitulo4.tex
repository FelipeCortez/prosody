% Capítulo 4 - Metodologia
% Implementação
% deve constar o instrumental, os métodos e as técnicas aplicados para a elaboração do trabalho;

\abrv[MBRPSOLA -- \emph{Multi-Band Resynthesis Pitch Synchronous Overlap and Add}]
\simb[ms (milissegundos)]
\simb[Hz (Hertz)]
\abrv[API -- \emph{Application Programming Interface}]
\abrv[JSON -- \emph{JavaScript Object Notation}]
\abrv[REST -- \emph{Representational State Transfer}]
\abrv[AJAX -- \emph{Asynchronous JavaScript And XML}]
\chapter{Editor de prosódia}

\section{Implementação}

\subsection{espeak-ng}
Foi utilizado o programa \emph{open-source} espeak-ng \cite{espeakng} para
realizar a normalização de texto e realizar a conversão grafema-fone, ou seja,
obter a partir do texto de entrada uma representação em fones. A saída gerada
é passada para o programa desenvolvido neste trabalho através da biblioteca
\code{subprocess} da linguagem Python. O comando utilizado para obter os grafemas pode ser visto no código \ref{lst:espeak}.

\begin{itemize}
\item A \emph{flag} \code{-v} seleciona uma voz. Neste caso, \code{pt-br}
\item \code{-x} e \code{-q} escrevem fones na saída em vez da fala sintetizada
\item \code{-sep=/} separa fones utilizando o caractere \code{/}
\end{itemize}

\begin{lstlisting}[caption=Utilização do programa espeak e saída correspondente,
  label=lst:espeak]
  $ espeak-ng -v pt-br 'Bom dia' -x -q --sep=/
  $ b/'o/N dZ/'i/&
\end{lstlisting}

Um fone precedido pelo caractere \code{\'} indica emph{primary stress}.

Apesar da existência de outras ferramentas para \emph{front end} para o
português brasileiro, optamos por esta pela facilidade de instalação. O sistema
TTS desenvolvido por \citeonline{couto} não foi encontrado para \emph{download}
no começo do desenvolvimento deste trabalho.

\subsection{MBROLA}
Cada voz do MBROLA tem suporte a um conjunto específico de dífonos. O programa
\code{sampa_mbrola.py} foi desenvolvido para converter a saída gerada em fones
suportados pela voz do MBROLA utilizada. O equivalente ao fone \code{&} gerado
pelo espeak é \code{a} na voz do MBROLA, por exemplo. A ferramenta de conversão
lê a tabela do arquivo \code{sampa_mbrola.tbl}, e substitui cada fone pelo
equivalente corrigido. A tabela possui três colunas: o fone que o espeak
produz, o fone equivalente para o MBROLA e um exemplo numa palavra do português.
Em \ref{sampambrola} é possível ver algumas linhas do arquivo.

\begin{lstlisting}[caption=Linhas da tabela de conversão, label=sampambrola]
  &  |  a  |  v_a_le
  6  |  @  |  tam_a_nho
  n  |  n  |  _n_unca
\end{lstlisting}

As durações padrão para cada fone estão contidas na tabela do arquivo
\code{durations.tbl}, adaptado da tabela utilizada pelo LianeTTS \cite{lianetts}. Cada
linha contém o fone e uma duração em milissegundos.

\subsection{INTSINT}
De acordo com \cite{intsintalg}

\begin{table}[htb]
	\textsf{\caption{Regras para INTSINT}}
	\center
	{
		\begin{tabular}{l|l}
			\hline
			Regra   & Cálculo \\
			\hline
            T   & $ \text{key} \times \sqrt{2^{range}} $ \\
            M   & $ \text{key} $ \\
            B   & $ \text{key} / \sqrt{2^{range}} $ \\
			\hline
            H   & $ \sqrt{P_{i - 1} \times T} $ \\
            U   & $ \sqrt{P_{i - 1} \times  \sqrt{P_{i - 1} \times  T}} $ \\
            S   & $ P_{i - 1} $ \\
            D   & $ \sqrt{P_{i - 1} \times  \sqrt{P_{i - 1} \times  B}} $ \\
            L   & $ \sqrt{P_{i - 1} \times B} $ \\
			\hline
		\end{tabular}
	}
	\label{tab:intsint}
\end{table}

\subsection{Arquitetura}
Foi utilizada uma arquitetura cliente-servidor REST \cite{rest} comumente observada em aplicações \emph{web} atuais. A interface gráfica consome uma API REST. A maneira como os
módulos se comunicam pode ser vista na figura \ref{fig:arch}.

\begin{figure}[!htbp]
\centering
\scalebox{0.80}{
    \begin{tikzpicture}[auto, >={Latex[inset=0pt, length=3mm, angle'=28,round]}, box/.style={draw,rounded corners,text width=3.0cm,align=center}]
    \node[] (txt) {Texto};
    \node[box, right=of txt] (esp)
        {eSpeakNG};
    \node[box, right=of esp] (con)
        {eSpeakNG para MBROLA};

    \node[box, below=of con] (ser)
        {Servidor};

    \node[box, left=of ser] (mbr)
        {MBROLA};
        
    \node[box, below=of ser] (int)
        {Interface gráfica};

    \draw[->] (txt) -- (esp);
    \draw[->] (esp) -- (con);

    \draw[->] (con) -- (ser);
    \draw[->] (mbr) -- (ser);

    \draw[<->] (ser) -- (int);

    \end{tikzpicture}
}
\caption{Arquitetura do sistema desenvolvido}
\label{fig:arch}
\end{figure}

\subsection{Módulo de prosódia}

O programa foi codificado em Python em sua versão 3.6. Pega resultado do espeak-ng, processa com editor gráfico e gera MBROLA.

Para alterar a prosódia manualmente, foi desenvolvido um editor gráfico
para \emph{web} utilizando HTML, CSS e JavaScript. A duração e altura de cada
fone pode ser especificado arrastando barras de controle. O editor se comunica
com o espeak-ng e MBROLA através de um servidor programado em Python utilizando
o \emph{framework} Flask para prover \emph{endpoints} de uma API REST.

\subsection{Endpoints}
Foram criados dois endpoints para a API, tornando possível a comunicação entre
interface gráfica e servidor.

\paragraph{[POST] /api/espeak} Recebe um texto como entrada e gera uma resposta
JSON com lista de fones. Um exemplo de resposta pode ser visto no código \ref{espeakpost}
\paragraph{[POST] /api/mbrola} Recebe uma lista de fones no formato MBROLA
descrito na seção \ref{sec:mbrola} e gera uma resposta com o nome do arquivo de
áudio contendo a fala sintetizada

\subsection{Editor gráfico}
Editor gráfico utilizando o \emph{framework} Vue.js. Comunica-se com o servidor
através de \emph{AJAX}

% \begin{figure}[!htbp]
% \centering
% \scalebox{0.65}{
%     \begin{tikzpicture}[auto, >={Latex[inset=0pt, length=3mm, angle'=28,round]}, box/.style={draw,rounded corners,text width=3.0cm,align=center}]
%     \node[] (txt) {Texto};
%     \node[box, right=0.7cm of txt] (pre)
%         {Pré-processador};
%     \node[box, right=0.2cm of pre] (mor)
%         {Analisador morfológico};
%     \node[box, right=0.2cm of mor] (con)
%         {Analisador de contexto};
%     \node[box, right=of con] (let)
%         {Módulo letra-som};
%     \node[box, right=of let] (pro)
%         {Gerador de prosódia};
%         
%     \draw[->] (txt) -- (pre);
%         
%     \node[box, fit=(pre)(mor)(con), label={[name=morfos_l] Analisador morfossintático}] (morfos) {};
%     \node[box, fit=(let)] (letsom) {};
%     \node[box, fit=(pro)] (proger) {};
%     
%     \node[inner sep=0, fit=(morfos)(letsom)(proger)] (all) {};
%     \node[box,
%           inner sep=0,
%           yshift=-0.75cm,
%           fit=(morfos.west)(proger.east),
%           label=center:Estrutura de dados,
%           minimum height=1cm,
%           below of=all]
%         (dad) {};
%     
%     \node[box, fit=(morfos)(morfos_l)(letsom)(proger)(dad), label=Processamento de linguagem natural] (nlp) {};
%     
%     \node[right=of dad, align=left] (out) {\emph{phones}\\ prosódia};
%     
%     \draw[->] (pre) -- (pre |- dad.north);
%     \draw[<->] (mor) -- (mor |- dad.north);
%     \draw[<->] (con) -- (con |- dad.north);
%     \draw[<->] (let) -- (let |- dad.north);
%     \draw[<->] (pro) -- (pro |- dad.north);
%     
%     \draw[->] (dad) -- (out);
%         
%     \end{tikzpicture}
% }
% \caption{Arquitetura de um sistema de geração de prosódia}
% \label{fig:nlpdiagram}
% \end{figure}
