% Capítulo 4 - Metodologia
% Implementação
% deve constar o instrumental, os métodos e as técnicas aplicados para a elaboração do trabalho;

\abrv[MBRPSOLA -- \emph{Multi-Band Resynthesis Pitch Synchronous Overlap and Add}]
\simb[ms (milissegundos)]
\simb[Hz (Hertz)]
\abrv[API -- \emph{Application Programming Interface}]
\abrv[JSON -- \emph{JavaScript Object Notation}]
\abrv[REST -- \emph{Representational State Transfer}]
\abrv[AJAX -- \emph{Asynchronous JavaScript And XML}]

\chapter{Editor de prosódia}

\section{Implementação}
\subsection{espeak-ng}
Foi utilizado o programa \emph{open-source} eSpeakNG \cite{espeakng} para
realizar as primeiras etapas do \emph{back end}, normalização de texto e
conversão grafema-fone, ou seja, obter a partir do texto de entrada uma
representação em fones. A saída gerada pelo programa é subsequentemente passada
para o programa desenvolvido neste trabalho através da biblioteca
\code{subprocess} da linguagem Python. O comando utilizado para obter os
grafemas pode ser visto no código
\ref{lst:espeak}.

\begin{itemize}
\item A \emph{flag} \code{-v} seleciona uma voz. Neste caso, \code{pt-br}
\item \code{-x} e \code{-q} escrevem fones na saída em vez da fala sintetizada.
\item \code{-sep=/} separa fones utilizando o caractere \code{/}
\end{itemize}

\begin{lstlisting}[caption=Utilização do programa espeak e saída correspondente,
  label=lst:espeak]
  $ espeak-ng -v pt-br 'Bom dia' -x -q --sep=/
  $ b/'o/N dZ/'i/&
\end{lstlisting}

Um fone precedido pelo caractere \code{\'} indica \emph{primary stress}.

Apesar da existência de outras ferramentas para \emph{front end} para o
português brasileiro, optamos pelo eSpeakNG pela disponibilidade e facilidade de
instalação. O sistema TTS desenvolvido por \citeonline{couto} não foi encontrado
para \emph{download} no início de desenvolvimento deste trabalho.

% MBROLA usa interpolação linear, INTSINT sugere quadratic spline

\subsection{MBROLA}
Cada voz do MBROLA tem suporte a um conjunto específico de dífonos. O programa
\code{sampa_mbrola.py} foi desenvolvido para converter a saída gerada em fones
suportados pela voz do MBROLA utilizada. O equivalente ao fone \code{/&/} gerado
pelo espeak é \code{/a/} na voz do MBROLA, por exemplo. A ferramenta de conversão
lê a tabela do arquivo \code{sampa_mbrola.tbl}, e substitui cada fone em notação
XSAMPA pelo equivalente. A tabela possui três colunas: o fone que o espeak
produz, o fone equivalente para o MBROLA e um exemplo numa palavra do português.
No código \ref{sampambrola} é possível ver algumas linhas do arquivo.

\begin{lstlisting}[caption=Extrato de linhas da tabela de conversão, label=sampambrola]
  b  |  b  | _b_arco
  k  |  k  | _c_om
  d  |  d  | _d_ose
  &  |  a  |  v_a_le
  6  |  @  |  tam_a_nho
  n  |  n  |  _n_unca
\end{lstlisting}

As durações padrão para cada fone estão contidas no arquivo
\code{durations.tbl}, adaptado da tabela utilizada pelo LianeTTS
\cite{lianetts}. Cada linha contém o fone e uma duração em milissegundos.

\subsection{INTSINT}
\label{intsintrules}
Como visto na seção \ref{intsintsec}, o INTSINT serve não apenas para análise,
mas também para síntese de contornos f0. Escolhemos utilizá-lo na implementação
pela simplicidade da notação e flexibilidade.

% \subsection{Módulo de prosódia}
% O programa foi codificado utilizando a linguagem de programação Python 3.6,
% calculando contorno f0 a partir de notação especificada pelo usuário seguindo o
% modelo INTSINT. O módulo é uma interface entre os programas eSpeakNG e MBROLA,
% calculando parâmetros a partir da representação textual inicial e sua conversão
% em fones geradas pelo \emph{front end}, calculando valores de duração para cada
% fone a partir de uma tabela e gerando frequências a partir do algoritmo
% encontrado na seção \ref{intsintrules}.

\subsection{Editor gráfico}
Para permitir refinamento de prosódia após cálculo de parâmetros, foi
desenvolvido um editor gráfico para \emph{web} utilizando HTML, CSS e
JavaScript. A duração e altura de cada fone pode ser especificado arrastando
barras de controle.
\subsubsection{Arquitetura}
 O editor se comunica com o espeak-ng e MBROLA através de um
servidor programado em Python utilizando o \emph{framework} Flask para prover
\emph{endpoints} de uma API \emph{RESTful}. Foi utilizada uma arquitetura cliente-servidor \cite{rest} comumente observada em aplicações \emph{web} atuais. A maneira como os módulos se comunicam pode ser vista na figura \ref{fig:arch}.

\begin{figure}[!htbp]
\centering
\scalebox{0.80}{
    \begin{tikzpicture}[auto, >={Latex[inset=0pt, length=3mm, angle'=28,round]}, box/.style={draw,rounded corners,text width=3.0cm,align=center}]
    \node[] (txt) {Texto};
    \node[box, right=of txt] (esp)
        {eSpeakNG};
    \node[box, right=of esp] (con)
        {eSpeakNG para MBROLA};

    \node[box, below=of con] (ser)
        {Servidor};

    \node[box, left=of ser] (mbr)
        {MBROLA};
        
    \node[box, below=of ser] (int)
        {Interface gráfica};

    \draw[->] (txt) -- (esp);
    \draw[->] (esp) -- (con);

    \draw[->] (con) -- (ser);
    \draw[->] (mbr) -- (ser);

    \draw[<->] (ser) -- (int);

    \end{tikzpicture}
}
\caption{Arquitetura para o programa da interface gráfica}
\label{fig:arch}
\end{figure}

\subsubsection{\emph{Endpoints}}
Foram criados dois \emph{endpoints} para a API, possibilitando a comunicação entre
interface gráfica e servidor.

\paragraph{[POST] /api/espeak} Recebe um texto como entrada e gera uma resposta
no formato JSON com lista de fones. Cada fone, por sua vez, possui campos \code{duration},
\code{phone_mbrola}, \code{phone_sampa} e \code{pitch_changes}. Um exemplo de resposta pode ser visto no código \ref{espeakpost}.
\paragraph{[POST] /api/mbrola} Recebe uma lista de fones no formato MBROLA
descrito na seção \ref{sec:mbrola} e gera uma resposta com o nome do arquivo de
áudio contendo a fala sintetizada.
\subsection{Utilização}
O programa pode ser executado através de uma interface de linha de comandos

\begin{lstlisting}[caption=Utilização por linha de comandos, label=cmdline, language=Python]
$ echo "Testando" | python3 sampa_mbrola.py > out.pho
$ mbrola ../br3/br3 out.pho out.wav; afplay out.wav
\end{lstlisting}

% A interface gráfica é uma aplicação \emph{web}
% Também foi desenvolvido um editor gráfico para permitir refinamento do contorno melódico
% gerado pelo algoritmo de \citeonline{intsintalg}.

% \begin{figure}[!htbp]
% \centering
% \scalebox{0.65}{
%     \begin{tikzpicture}[auto, >={Latex[inset=0pt, length=3mm, angle'=28,round]}, box/.style={draw,rounded corners,text width=3.0cm,align=center}]
%     \node[] (txt) {Texto};
%     \node[box, right=0.7cm of txt] (pre)
%         {Pré-processador};
%     \node[box, right=0.2cm of pre] (mor)
%         {Analisador morfológico};
%     \node[box, right=0.2cm of mor] (con)
%         {Analisador de contexto};
%     \node[box, right=of con] (let)
%         {Módulo letra-som};
%     \node[box, right=of let] (pro)
%         {Gerador de prosódia};
%         
%     \draw[->] (txt) -- (pre);
%         
%     \node[box, fit=(pre)(mor)(con), label={[name=morfos_l] Analisador morfossintático}] (morfos) {};
%     \node[box, fit=(let)] (letsom) {};
%     \node[box, fit=(pro)] (proger) {};
%     
%     \node[inner sep=0, fit=(morfos)(letsom)(proger)] (all) {};
%     \node[box,
%           inner sep=0,
%           yshift=-0.75cm,
%           fit=(morfos.west)(proger.east),
%           label=center:Estrutura de dados,
%           minimum height=1cm,
%           below of=all]
%         (dad) {};
%     
%     \node[box, fit=(morfos)(morfos_l)(letsom)(proger)(dad), label=Processamento de linguagem natural] (nlp) {};
%     
%     \node[right=of dad, align=left] (out) {\emph{phones}\\ prosódia};
%     
%     \draw[->] (pre) -- (pre |- dad.north);
%     \draw[<->] (mor) -- (mor |- dad.north);
%     \draw[<->] (con) -- (con |- dad.north);
%     \draw[<->] (let) -- (let |- dad.north);
%     \draw[<->] (pro) -- (pro |- dad.north);
%     
%     \draw[->] (dad) -- (out);
%         
%     \end{tikzpicture}
% }
% \caption{Arquitetura de um sistema de geração de prosódia}
% \label{fig:nlpdiagram}
% \end{figure}
