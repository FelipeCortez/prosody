% Capítulo 3 - Referencial ou embasamento teórico
% Revisão da literatura
% texto no qual se deve apresentar os aspectos teóricos, isto é, os conceitos utilizados e a definição dos mesmos; nesta parte faz-se a revisão de literatura sobre o assunto, resumindo-se os resultados de estudos feitos por outros autores, cujas obras citadas e consultadas devem constar nas referências;

\simb[\% (fronteira de enunciado para ToBI)]
\abrv[INTSINT -- \emph{International Transcription System for Intonation}]
\abrv[HMM -- \emph{Hidden Markov Model}]

\chapter{Revisão da literatura}

% Intonational phrase: sintagmas entoacionais
% Tone boundary: fronteira prosódica?
% Pitch accent: acento de pitch

% https://www.ime.usp.br/~cpaz/downloads/algorithm-portuguese.pdf

\subsection{Abordagens}
\subsubsection{Klaat}
Síntese por formantes. \cite{espeakng} usa uma mistura do algoritmo de Klatt com
sons de consoantes pré-gravados.
\subsubsection{Unit selection e dífonos}
Abordagem utilizada pelo programa MBROLA. Consiste em gravar fala, separar
pedaços de dois em dois.

Algumas vozes para o MaryTTS \cite{marytts} utilizam HMMs, isto é, Modelos
ocultos de Markov para \emph{unit selection}. Há, inclusive, uma voz brasileira
feita a partir de HMMs: \cite{couto}.
\subsubsection{\emph{Deep Neural Networks}}
Projetos mais recentes como \cite{merlin} utilizam redes neurais para estimação de
parâmetros no \emph{back-end}.

% Resultados realistas, mas não há como controlar parâmetros: \cite{merlin}

\section{Sistemas TTS}
\subsubsection{MBROLA}
Baseado em síntese por dífonos. Antigo mas qualidade decente. Três vozes para o português brasileiro.

\section{Sistemas TTS com suporte a português}
\subsection{MaryTTS}
Projeto FalaBrasil \cite{falabrasil}, \cite{couto}
\subsubsection{Grafema-fone}
Para o português brasileiro, foram encontrados os conversores
da USP: \cite{g2pusp} do projeto falabrasil: \cite{falabrasil}.
\subsection{LianeTTS}
Projeto da SERPRO LianeTTS (MBROLA)
\subsection{espeak}
Projeto do Dunn.

\subsection{Prosódia no português brasileiro}
Trabalhos de Moraes, Tenani, ...
\subsection{Modelos de prosódia}
\subsubsection{Modelo autossegmental e métrico}
Modelo autossegmental e métrico: Pierrehumbert, Moraes (pitch analysis by synthesis).
ref Moraes, Intonation Systems (20 languages).
\subsubsection{IPO}
% página 32
\cite{ipo} analisa a prosódia para o português brasileiro através do modelo IPO,
seguindo crítica de Lucente que o sistema ToBI é inapropriado.
 % "esse sistema de notação não abarca certas características fonéticas das curvas de F0 que perceptivamente são relevantes para o reconhecimento desses contornos"
\subsubsection{INTSINT}
INTSINT é um sistema de anotação para prosódia.
% The AM model is phonological, the INTSINT model phonetic and the Fujisaki and Tilt models acoustic''