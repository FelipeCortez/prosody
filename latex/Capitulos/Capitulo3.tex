% Capítulo 3 - Referencial ou embasamento teórico
% Revisão da literatura
% texto no qual se deve apresentar os aspectos teóricos, isto é, os conceitos utilizados e a definição dos mesmos; nesta parte faz-se a revisão de literatura sobre o assunto, resumindo-se os resultados de estudos feitos por outros autores, cujas obras citadas e consultadas devem constar nas referências;

% \simb[\% (fronteira de enunciado para ToBI)]
\simb[f0 (frequência fundamental da fala)]
\abrv[INTSINT -- \emph{International Transcription System for Intonation}]
\abrv[HMM -- \emph{Hidden Markov Model}]
\abrv[SSML -- \emph{Speech Synthesis Markup Language}]
\abrv[XML -- \emph{eXtensible Markup Language}]
\abrv[W3C -- \emph{World Wide Web Consortium}]

\chapter{Revisão da literatura}

% Resultados realistas, mas não há como controlar parâmetros: \cite{merlin}

\section{Sistemas TTS para o português brasileiro}
\label{sistemas}

Destacamos aqui sistemas TTS desenvolvidos por ordem cronológica a fim de
evidenciar a evolução das tecnologias utilizadas e as tendências para trabalhos futuros.
\paragraph{Aiuruetê \cite{aiuruete}}
Desenvolvido com o objetivo de obter um sistema TTS com fala natural sem custo
computacional elevado. Descartam síntese articulatória e formantes porque o
custo seria muito alto, como descrito na seção \ref{backend}, optando por uma
solução concatenativa. Para a prosódia, a parte de \emph{pitch} utiliza curvas
f0 associadas a frases declarativas com declinação e as durações segmentais são
obtidas por redes neurais treinadas a partir de parâmetros como número de vogais e clíticos.

\paragraph{eSpeakNG \cite{espeakng}}
Projeto \emph{open-source} com suporte a múltiplas linguagens. É modular,
permitindo \emph{back end} com síntese por formantes ou concatenativa. Também
possibilita que seja utilizado apenas o \emph{front end}, gerando representação
no formato X-SAMPA (\emph{Extended Speech Assessment Methods Phonetic
  Alphabet}). No seu módulo de prosódia, determina contorno f0 a partir de
pontuação e tabela com regras para \emph{pre-head}, \emph{head}, \emph{nucleus}
e \emph{tail}. As durações são fixas para fones diferentes e também são
determinadas por uma tabela.

\paragraph{Microsoft \cite{hmmmicrosoft}}
Desenvolvido pela Microsoft, realiza a marcação de ênfase e separação silábica
utilizando um dicionário. Destaca-se por utilizar um corpus composto por 2000
sintagmas foneticamente anotados para determinar prosódia probabilisticamente.

\paragraph{\cite{couto}}
Foi desenvolvido com base no \emph{framework} MaryTTS um sistema completo para
português brasileiro baseado em HMMs. O projeto iniciou com uma implementação
de um módulo de conversão grafema-fone LaPS-G2P \cite{g2pusp}, tornando-se então
um sistema completo \cite{couto}, e foi subsequentemente estendido por
\cite{costa} para funcionar de maneira \emph{stand-alone}, isto é, ser utilizado
sem necessidade de instalação do \emph{framework} MaryTTS. A prosódia é gerada
pelos parâmetros das cadeias de Markov, ou seja, determinada probabilisticamente
a partir do corpus de treinamento.

% TTS systems evolved from a knowledge-based paradigm to  a  pragmatic
% data-driven approach.  With  respect  to  the technique  adopted  for  the
% back  end,  the  main  categories are  the  formant-based,  concatenative
% and,  more  recently, HMM-based [1], [7], [8]. \cite{couto} (ou é costa?)

\paragraph{LianeTTS \cite{lianetts}}
Projeto da SERPRO, o LianeTTS utiliza síntese concatenativa através do programa
MBROLA. Assim como o eSpeakNG, os componentes são baseados em regras e tabelas.
O cálculo de curva f0 é feito com base em \emph{parts-of-speech tagging}, ou
seja, atribuição de classes gramaticais a cada palavra do texto de entrada.

\paragraph{MBROLA}
\label{sec:mbrola}
MBROLA é uma ferramenta para geração de voz baseada em síntese concatenativa por
dífonos desenvolvida com o objetivo de fomentar pesquisas acadêmicas em geração
de prosódia \cite{mbrola}. É utilizada como \emph{back-end} para diversos
sistemas TTS, como MaryTTS, Festival \cite{festival} e eSpeakNG, e possui três vozes disponíveis para o português brasileiro. % \cite{mbrola} The ultimate goal is to boost up academic research on speech synthesis, and particularly on prosody generation, known as one of the biggest challenges taken up by Text-to-Speech synthesizers for the years to come.

\begin{lstlisting}[caption=Exemplo de arquivo de entrada para MBROLA]
      _ 150 50 150
      t  70 50 125
      e 125 50 75
      c  70 50 125
      e 125 50 75
      c  70 50 125
      e 116 20 232 80 300
      _ 150 50 150
\end{lstlisting}

\subsubsection{Formato}
Em cada linha, tem-se um fone ou um silêncio representado pelo \emph{underscore} seguido por uma duração em milissegundos e, por último, um ou mais pares de porcentagem e frequência em Hertz determinando alvos para a curva f0. Como exemplo, na penúltima linha temos o fone \/e\/ com duração de 116 ms e dois alvos para altura, 232 Hz em 20\% e 300 Hz em 80\%.

Cada voz gravada provê uma tabela com os fones que podem ser utilizados. Utilizamos neste trabalho a voz br3 desenvolvida por Denis R. Costa disponível no site oficial do projeto MBROLA.
% referência?

\subsection{Modelos de análise entoacional}
\subsubsection{Teoria métrica-autossegmental}
Utilizada por \citeonline{moraes2008} para analisar uma mesma frase com diferentes intenções.
% Enquanto a teoria MA, representada pelo sistema ToBI, se baseia em aspectos 
% lineares da estrutura tonal, na identifi cação dos 
% pitch accents
%  e no alinhamento abs-
% trato dessa estrutura com o material linguístico, o sistema DaTo de notação entoa-
% cional (LUCENTE, 2008; 2012) se concentra na convergência de aspectos fonéticos 
% – velocidade, intensidade, altura, duração – da curva entoacional a fi m de atingir um 
% alvo ou desempenhar uma tarefa linguística por meio dos contornos entoacionais, da 
% gama de variação tonal e do alinhamento específi co com o material linguístico

% O sistema DaTo foi desenvolvido com base na entoação do português bra-
% sileiro (PB) e trabalha com o conceito de contorno dinâmico, que é defi nido em 
% Lucente (2012, p. 99) como “uma unidade tonal que contém elementos comuni-
% cativos expressos em uma trajetória ideal da curva entoacional, especifi cada por 
% um alvo a ser atingido e associada a uma unidade segmental linguística”
\subsubsection{IPO}
% página 32
Sistema proposto pela Escola Holandesa. Utilizada por \citeonline{ipo} para analisar a
prosódia no português brasileiro.
\subsubsection{INTSINT}
Key (frequência base)
Range (intervalo entre ponto mais alto e mais baixo do f0).
INTSINT é um sistema de anotação para prosódia. Top (T), Mid(M), Bottom (B).
Higher (H), Same (S), Lower(L). Acentos: Upstepped (S), Downstepped (D).

A tabela \ref{tab:intsint} detalha como cada marcação é convertida para uma
frequência em Hertz para determinação de curva f0.

Utilizado por \citeonline{intsintpt} e \citeonline{moraes1998} para analisar entoação para o
português brasileiro.

% The INTSINT model of Hirst et al [212], [213], [211], was deve loped in an attempt to provide a comprehensive  and multi-lingual  transcription  system fo r intonation.   The model can be seen as “theory-neutral” in that it was designed to transcribe th e intonation of utterances as a way of annotating databases and thereby providing the raw data upo n which intonational theories could be developed. Hirst has described the development of INTSINT a s an attempt to design an equivalent of IPA for intonation. As stated in Section 6.10, there is no s uch thing as completely theory neutral model as all models make some assumptions.  Nevertheless, IN TSINT certainly fulfills its main goals of allowing a phonetic transcription of an utterance t o be made without necessarily deciding which theory or model of intonation will be subsequently use d. INTSINT describes an utterance’s intonation by a sequence o f labels each of which repre- sents a target point. These target points are defined either b y reference to the speaker’s pitch range, 242 Chapter 9. Synthesis of Prosody in which case they are marked Top (T), Mid (M) or Bottom (B), or by reference to the previous tar- get point, in which case they are marked Higher (H), Same (S) o r Lower (L). In addition, accents can be marked as Upstepped (S) or Downstepped (D). Hirst [212] describes this system in detail and shows how it c an be applied to all the major languages.  Several algorithms have also been developed for extracting the labels automatically from the acoustics and for synthesizing F0 contours from the labels 2 .  Applications of this model to synthesis include Veronis et al [475]
% The AM model is phonological, the INTSINT model phonetic and the Fujisaki and Tilt models acoustic''
\subsubsection{DaTo (\emph{Dynamic Tones})}
Apesar de o modelo ToBI para anotação entoacional ter sido utilizado para analisar o
português brasileiro em diversos trabalhos, \citeonline{lucentetobi} argumenta que
há características perceptíveis que a notação não consegue expressar, propondo o
modelo DaTo com base na entonação do português brasileiro.

\subsection{Prosódia afetiva em sistemas TTS}
\label{prosafe}
\subsubsection{SSML}
A linguagem de marcação \emph{Speech Synthesis Markup Language} foi criada
motivada pela dificuldade de predição computacional de pronúncia
\cite{ssmlpaper}. Quando originalmente proposta, diferentes sistemas TTS
permitiam anotações extra-textuais para auxiliar a estimação de parâmetros, mas
usuários tinham que aprender um sistema de anotação para cada programa
diferente. A proposta da SSML é que os sistemas TTS recebam o texto anotado numa
linguagem unificada. A linguagem foi adotada por soluções \emph{open-source}
como MaryTTS e espeak-ng, além dos sistemas proprietários encontrados em Alexa,
Google Assistant e Cortana. A especificação é mantida pela W3C \cite{ssml}.
A especificação cita os elementos \emph{emphasis}, \emph{break} e \emph{prosody}
como marcadores que podem auxiliar o processador de linguagem natural a gerar
parâmetros prosódicos apropriados.

\begin{lstlisting}[caption=Exemplo de texto anotado com SSML]
<speak>
    Siga
    <emphasis level="strong">aquele</emphasis> 
    carro.
</speak> 
\end{lstlisting}

% http://mary.dfki.de/documentation/overview.html
% https://www.w3.org/TR/emotionml/
\subsubsection{EmotionML}
EmotionML \cite{emotionml} foi criada para várias coisas, uma delas é ajudar
algoritmos a determinarem prosódia. \cite{emotionmary} descreve um
\emph{framework} para implementação de determinação de prosódia a partir de
anotações em EmotionML utilizando o \emph{framework} MaryTTS.

Como explicitado por \cite{taylor2009}, não há um acordo quanto ao sistema mais
apropriado para representar emoções. A linguagem de marcação tem suporte a
múltiplos sistemas descritivos, como categorias, dimensões, appraisals e action
tendencies. As categorias podem receber valores discretos (como pode ser visto
no código \ref{lst:discreto}), determinando se uma emoção está presente ou valores
contínuos, determinando a intensidade de uma emoção específica (como pode ser
visto na figura \ref{lst:ssmlemotion}), permitindo
múltiplas categorias simultaneamente.

% Emotions can be represented in terms of four types of de-
% scriptions  taken  from  the  scientific  literature:  categories,
% dimensions, appraisals, and action tendencie

\begin{lstlisting}[caption=Exemplo de texto anotado com EmotionML com parâmetros
  discretos, label=lst:discreto]
<emotionml version="1.0" xmlns="http://www.w3.org/2009/10/emotionml">
  <emotion category-set="http://www.w3.org/TR/emotion-voc/xml#everyday-categories">
  <emotion>
    <category name="happy" />
    Que bom te ver!
  </emotion>
</emotionml>
\end{lstlisting}

\begin{lstlisting}[caption=Exemplo de texto anotado com SSML e EmotionML
  (adaptado de \cite{emotionml}), label=lst:ssmlemotion]
<speak version="1.1" xmlns="http://www.w3.org/2001/10/synthesis"
         xmlns:emo="http://www.w3.org/2009/10/emotionml"
         xml:lang="en-US">
    <s>
        <emo:emotion category-set="http://www.w3.org/TR/emotion-voc/xml#everyday-categories">
            <emo:category name="worried" value="0.4"/>
        </emo:emotion>
        Precisa de ajuda?
    </s>
</speak>
\end{lstlisting}

% Expressive speech synthesis, generating synthetic speech with different emotions, such as happy or sad, friendly or apologetic; expressive synthetic speech would for example make more information available to blind and partially sighted people, and enrich their experience of the content;

\subsubsection{Anotação manual}
Uma solução mais simples para geração de contornos f0 para prosódia efetiva é
permitir que o usuário especifique na entrada do programa TTS marcações
prosódicas utilizando um dos modelos de análise entoacional vistos
anteriormente. Apesar de requerer conhecimento de fonologia, é uma opção 
viável enquanto não são desenvolvidos algoritmos para determinação de prosódia a
partir de marcação emocional. O sistema Festival \cite{festival} disponibiliza uma maneira de
especificar entoação seguindo o modelo ToBI, como pode ser visto na tabela
\ref{tobifest}.

\begin{lstlisting}[caption=Anotações no modelo ToBI para o sistema TTS Festival,
  label=tobifest]
(Utterance Words 
 (The
  (boy ((accent L*)))
  saw
  the
  (girl ((accent H*) (tone L-)))
  with 
  the
  (telescope ((accent H*) (tone H-H%))))))
\end{lstlisting}