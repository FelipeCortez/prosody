% Capítulo 3 - Referencial ou embasamento teórico
% Revisão da literatura
% texto no qual se deve apresentar os aspectos teóricos, isto é, os conceitos utilizados e a definição dos mesmos; nesta parte faz-se a revisão de literatura sobre o assunto, resumindo-se os resultados de estudos feitos por outros autores, cujas obras citadas e consultadas devem constar nas referências;

\simb[\% (fronteira de enunciado para ToBI)]
\abrv[INTSINT -- International Transcription System for Intonation]
\abrv[HMM -- \emph{Hidden Markov Model}]

\chapter{Revisão da literatura}

% Intonational phrase: sintagmas entoacionais
% Tone boundary: fronteira prosódica?
% Pitch accent: acento de pitch

% https://www.ime.usp.br/~cpaz/downloads/algorithm-portuguese.pdf

\section{Sistemas TTS}
Um sistema \emph{text-to-speech}
% \subsection{Breve história?}
% Modelos físicos, Bell Labs.
\subsection{Estrutura}
Normalização, conversão grafema-fone, regras prosódicas (front-end).
Síntese de voz (back-end).
\subsection{Normalização de texto}
A primeira parte do processo de conversão de texto para fala é transformar a entrada em grafemas, ou seja.
\subsection{Conversão grafema-fone}
Para o português brasileiro, foram encontrados os conversores
da USP: \cite{g2pusp}
do projeto falabrasil: \cite{falabrasil}.
\subsection{Fonemas e fones}
\subsection{Abordagens}
\subsubsection{Klaat}
Síntese por formantes. \cite{espeakng} usa uma mistura do algoritmo de Klatt com
sons de consoantes pré-gravados.
\subsubsection{Unit selection e dífonos}
Abordagem utilizada pelo programa MBROLA. Consiste em gravar fala, separar
pedaços de dois em dois.
\subsubsection{Hidden Markov Models}
Algumas vozes para o MaryTTS \cite{marytts} utilizam HMMs, isto é, Modelos ocultos de Markov.
\subsubsection{DNN}
% Resultados realistas, mas não há como controlar parâmetros: \cite{merlin}
\subsection{TTS em português}
LianeTTS (MBROLA), HMM-based \cite{couto}, MaryTTS (FalaBrasil) \cite{falabrasil}.
\section{Prosódia}
\subsection{Componentes}
\citeonline{taylor2009} mostra três tipos de prosódia.
Stress (loudness and phonatory force)
Syllabic length
F0, intensidade, duração
Intonational tune
Downdrift
Microprosódia
\subsection{Tipos de prosódia}
\subsubsection{Aumentativa}
\subsubsection{Suprasegmental}
\subsubsection{Afetiva}
\subsection{Prosódia como elemento extra-textual}
Justifica abordagem do trabalho: considerando o texto como sequência de
palavras, é difícil determinar prosódia afetiva. Gerar a prosódia certa é uma
questão de Natural Language Understanding, isto é, é preciso entender o texto
para gerar os contornos melódicos afetivos.
\subsection{Prosódia no português brasileiro}
Trabalhos de Moraes, Tenani, ...
\subsection{Modelos de prosódia}
\subsubsection{IPO}
\cite{ipo} analisa segundo o modelo IPO.
\subsubsection{Modelo autossegmental e métrico}
Modelo autossegmental e métrico: Pierrehumbert, Moraes (pitch analysis by synthesis).
ref Moraes, Intonation Systems (20 languages).
\subsubsection{INTSINT}
INTSINT é um sistema de anotação para prosódia.
% The AM model is phonological, the INTSINT model phonetic and the Fujisaki and Tilt models acoustic''
\subsection{Prosódia em sistemas TTS}
\subsubsection{SSML}
SSML \cite{SSML}, especificação mantida pela W3C.
% https://www.w3.org/TR/emotionml/
MaryTTS usa ToBI, MBROLA, Unit selection do FreeTTS, SSML.
Alexa, Google Assistant e Cortana têm suporte a SSML.
% http://mary.dfki.de/documentation/overview.html

% \subsection{Trabalhos semelhantes?}
