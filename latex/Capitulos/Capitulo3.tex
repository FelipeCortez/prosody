% Capítulo 3 - Referencial ou embasamento teórico
% Revisão da literatura
% texto no qual se deve apresentar os aspectos teóricos, isto é, os conceitos utilizados e a definição dos mesmos; nesta parte faz-se a revisão de literatura sobre o assunto, resumindo-se os resultados de estudos feitos por outros autores, cujas obras citadas e consultadas devem constar nas referências;

\chapter{Capítulo 3}

\simb[\% (fronteira de enunciado para ToBI)]{\%}

\cite{ipo} analisa segundo o modelo IPO.

O \abrv[INTSINT -- International Transcription System for Intonation]{INTSINT} é...

% Intonational phrase: sintagmas entoacionais
% Tone boundary: fronteira prosódica?
% Pitch accent: acento de pitch

\section{TTS}
Um sistema \emph{text-to-speech}
% \subsection{Breve história?}
% Modelos físicos, Bell Labs.
\subsection{Fonemas e fones}
\subsection{Abordagens}
\subsubsection{Klaat}
Síntese por formantes. \cite{espeakng} usa uma mistura do algoritmo de Klatt com
sons de consoantes pré-gravados.
\subsubsection{Unit selection e dífonos}
Abordagem utilizada pelo programa MBROLA. Consiste em gravar fala, separar
pedaços de dois em dois.
\subsubsection{Hidden Markov Models}
Algumas vozes para o MaryTTS \cite{marytts} utilizam \abrv[HMM -- \emph{Hidden Markov Model}]{HMM}, isto é, Modelo oculto de Markov.
\subsubsection{DNN}
% Resultados realistas, mas não há como controlar parâmetros: \cite{merlin}
\subsection{TTS em português}
LianeTTS (MBROLA), HMM-based \cite{couto}, MaryTTS (FalaBrasil) \cite{falabrasil}.
\section{Prosódia}
\subsection{Componentes}
\citeonline{taylor2009} mostra três tipos de prosódia.
Stress (loudness and phonatory force)
Syllabic length
F0, intensidade, duração
Intonational tune
Downdrift
Microprosódia
\subsection{Tipos de prosódia}
\subsubsection{Aumentativa}
\subsubsection{Suprasegmental}
\subsubsection{Afetiva}
\subsection{Prosódia como elemento extra-textual}
Justifica abordagem do trabalho: considerando o texto como sequência de
palavras, é difícil determinar prosódia afetiva. Gerar a prosódia certa é uma
questão de Natural Language Understanding, isto é, é preciso entender o texto
para gerar os contornos melódicos afetivos.
\subsection{Prosódia no português brasileiro}
Trabalhos de Moraes, Tenani, ...
\subsection{Modelos de prosódia}
\subsubsection{IPO}
\subsubsection{Modelo autossegmental e métrico}
Modelo autossegmental e métrico: Pierrehumbert, Moraes (pitch analysis by synthesis).
ref Moraes, Intonation Systems (20 languages).
\subsubsection{INTSINT}
INTSINT: IPA para prosódia (mais ou menos o que eu quero fazer, mas INTSINT é
para análise).

British school, Fujisaki, Tilt
``The AM model is phonological, the INTSINT model phonetic and the Fujisaki and Tilt models acoustic''.
\subsection{Prosódia em sistemas TTS}
SSML \cite{SSML}, especificação mantida pela W3C.
\subsection{Trabalhos semelhantes?}