% Capítulo 3 - Referencial ou embasamento teórico
% Revisão da literatura
% texto no qual se deve apresentar os aspectos teóricos, isto é, os conceitos utilizados e a definição dos mesmos; nesta parte faz-se a revisão de literatura sobre o assunto, resumindo-se os resultados de estudos feitos por outros autores, cujas obras citadas e consultadas devem constar nas referências;

\chapter{Capítulo 3}


\section{Seção 1}

Teste de uma tabela:

\begin{table}[htb]
	% Título de tabelas sempre aparecem antes da tabela
	\textsf{\caption{Tabela sem sentido}}
	\center
	{
		\begin{tabular}{l|l}
			\hline
			Titulo Coluna 1   & Título Coluna 2\\
			\hline
			X                 & Y\\
			X                 & W\\
			\hline
		\end{tabular}
	}
	\label{tab:TabelaSemSentido}
\end{table}


\section{Seção 2}

Seção 2


\subsection{Subseção 2.1}

Referência à tabela definida no início: \ref{tab:TabelaSemSentido}


\subsection{Subseção 2.2}

Subsection 2.2


\section{Seção 3}

Seção 3
