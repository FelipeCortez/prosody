% Capítulo 3 - Referencial ou embasamento teórico
% Revisão da literatura
% texto no qual se deve apresentar os aspectos teóricos, isto é, os conceitos utilizados e a definição dos mesmos; nesta parte faz-se a revisão de literatura sobre o assunto, resumindo-se os resultados de estudos feitos por outros autores, cujas obras citadas e consultadas devem constar nas referências;

\simb[\% (fronteira de enunciado para ToBI)]
\abrv[INTSINT -- \emph{International Transcription System for Intonation}]
\abrv[HMM -- \emph{Hidden Markov Model}]

\chapter{Revisão da literatura}

% Intonational phrase: sintagmas entoacionais
% Tone boundary: fronteira prosódica?
% Pitch accent: acento de pitch

% https://www.ime.usp.br/~cpaz/downloads/algorithm-portuguese.pdf

\subsection{Abordagens}
\subsubsection{Klaat}
Síntese por formantes. \cite{espeakng} usa uma mistura do algoritmo de Klatt com
sons de consoantes pré-gravados.
\subsubsection{\emph{Unit selection} e dífonos}
Abordagem utilizada pelo programa MBROLA. Consiste em gravar fala, separar
pedaços de dois em dois.

\subsubsection{HTS}
% http://hts.sp.nitech.ac.jp/
Algumas vozes para o MaryTTS \cite{marytts} utilizam HMMs, isto é, Modelos
ocultos de Markov para \emph{unit selection}. Há, inclusive, uma voz brasileira
feita a partir de HMMs: \cite{couto}.
Projetos mais recentes como \cite{merlin,dnngoogle} utilizam redes neurais para
estimação de parâmetros. O trabalho da Google informa? que a estimação da curva
F0 é uma possível melhoria.

% Resultados realistas, mas não há como controlar parâmetros: \cite{merlin}

\section{Sistemas TTS}
\subsubsection{MBROLA}
Baseado em síntese por dífonos. Antigo mas qualidade decente. Três vozes para o
português brasileiro. \cite{mbrola} % \cite{mbrola} The ultimate goal is to boost up academic research on speech synthesis, and particularly on prosody generation, known as one of the biggest challenges taken up by Text-to-Speech synthesizers for the years to come.

\subsection{Prosódia em sistemas TTS}
\subsubsection{SSML}
SSML \cite{ssml}, especificação mantida pela W3C.
% https://www.w3.org/TR/emotionml/
MaryTTS usa ToBI, MBROLA, Unit selection do FreeTTS, SSML.
Alexa, Google Assistant e Cortana têm suporte a SSML.
SSML é apenas um modificador (insuficiente para estimar prosódia?)
% http://mary.dfki.de/documentation/overview.html
\subsubsection{EmotionML}
EmotionML \cite{emotionml} foi criada para várias coisas, uma delas

\section{Sistemas TTS com suporte a português}
\subsection{MaryTTS}
Projeto FalaBrasil \cite{falabrasil}, \cite{couto}
\subsubsection{Grafema-fone}
Para o português brasileiro, foram encontrados os conversores
da USP: \cite{g2pusp} do projeto falabrasil: \cite{falabrasil}.
\subsection{LianeTTS}
Projeto da SERPRO LianeTTS (MBROLA)
\subsection{espeak}
Projeto do Dunn.

\subsection{Prosódia no português brasileiro}
Trabalhos de Moraes, Tenani, ...
\subsection{Modelos de prosódia}
\subsubsection{Teoria métrica-autossegmental}
Modelo autossegmental e métrico: Pierrehumbert, Moraes (pitch analysis by synthesis).
ref Moraes, Intonation Systems (20 languages).
\subsubsection{DaTo}
% Enquanto a teoria MA, representada pelo sistema ToBI, se baseia em aspectos 
% lineares da estrutura tonal, na identifi cação dos 
% pitch accents
%  e no alinhamento abs-
% trato dessa estrutura com o material linguístico, o sistema DaTo de notação entoa-
% cional (LUCENTE, 2008; 2012) se concentra na convergência de aspectos fonéticos 
% – velocidade, intensidade, altura, duração – da curva entoacional a fi m de atingir um 
% alvo ou desempenhar uma tarefa linguística por meio dos contornos entoacionais, da 
% gama de variação tonal e do alinhamento específi co com o material linguístico

% O sistema DaTo foi desenvolvido com base na entoação do português bra-
% sileiro (PB) e trabalha com o conceito de contorno dinâmico, que é defi nido em 
% Lucente (2012, p. 99) como “uma unidade tonal que contém elementos comuni-
% cativos expressos em uma trajetória ideal da curva entoacional, especifi cada por 
% um alvo a ser atingido e associada a uma unidade segmental linguística”
\subsubsection{IPO}
% página 32
\cite{ipo} analisa a prosódia para o português brasileiro através do modelo IPO,
seguindo crítica de Lucente que o sistema ToBI é inapropriado.
 % "esse sistema de notação não abarca certas características fonéticas das curvas de F0 que perceptivamente são relevantes para o reconhecimento desses contornos"
\subsubsection{INTSINT}
INTSINT é um sistema de anotação para prosódia.
% The AM model is phonological, the INTSINT model phonetic and the Fujisaki and Tilt models acoustic''