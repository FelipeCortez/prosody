% Apêndice
\apendice
\chapter{Primeiro apêndice}

\begin{lstlisting}[caption=Servidor, label=servidor, language=Python]
import sqlite3
import subprocess
import sampa_mbrola
import secrets
from flask import Flask, g, jsonify, render_template, abort, request, send_file
from flask_cors import CORS

app = Flask(__name__)
CORS(app)

@app.route('/api/espeak', methods=['POST'])
def frontend():
    text = request.form['text']
    converter = sampa_mbrola.Converter()
    sentence = converter.convert_sentence(text)
    return jsonify(sentence.dictify())

@app.route('/api/gen_audio', methods=['POST'])
def gen_audio():
    text = request.form["text"]
    converter = sampa_mbrola.Converter()
    sentence = converter.convert_sentence(text)

    token = secrets.token_hex(nbytes=4)
    with open(f"out/{token}.pho", 'w') as f:
        f.write(str(sentence))

    mbrola_str = f"mbrola br3/br3 out/{token}.pho out/{token}.wav"
    p = subprocess.Popen(mbrola_str, stdout=subprocess.PIPE, shell=True)
    (output, err) = p.communicate()
    p_status = p.wait()

    return jsonify({"token": token, "sentence": sentence.dictify()})

@app.route('/api/get_audio/<token>', methods=['GET'])
def get_audio(token):
    return send_file(
        f"out/{token}.wav",
        mimetype="audio/wav",
        as_attachment=True,
        attachment_filename=f"out/{token}.wav")

if __name__ == '__main__':
    app.run(debug=True)
\end{lstlisting}

\begin{lstlisting}[caption=Conversor eSpeakNG-MBROLA, label=conversao, language=Python]
import sys
import re
import subprocess
import random
import json
import uuid
from math import sqrt
from collections import OrderedDict
from typing import List

def flatten(lst: list):
    return [item for sublist in lst for item in sublist]

class Phone():
    def __init__(
            self,
            phone_sampa: str,
            phone_mbrola: str,
            duration: int,
            pitch_changes: list,
            stress: bool = False
    ):
        self.phone_sampa = phone_sampa
        self.phone_mbrola = phone_mbrola
        self.duration = duration
        self.pitch_changes = pitch_changes
        self.stress = stress

    def as_line(self):
        return "{} {} {}".format(
            self.phone_mbrola,
            self.duration,
            " ".join([str(item) for item in flatten(self.pitch_changes)])
        )

class Sentence():
    def __init__(self, phones: List[Phone]=None):
        if phones is None:
            self.phones = []
        else:
            self.phones = phones

    def mbrola_phones(self):
        return [phone.phone_mbrola for phone in self.phones]

    def dictify(self):
        return [vars(phone) for phone in self.phones]

    def __repr__(self):
        return "\n".join([phone.as_line() for phone in self.phones])


class Converter():
    def __init__(self):
        self.load_sampa_mbrola()
        self.load_durations()

    def load_sampa_mbrola(self):
        equivs = {}
        with open("sampa_mbrola.tbl") as f:
            for line in f:
                k, v, _ = line.split()
                equivs[k] = v

        self.equivs = OrderedDict(
            sorted(equivs.items(), key=lambda t: -len(t[0])))

    def load_durations(self):
        durations = {}
        with open("durations.tbl") as f:
            for line in f:
                k, v = line.split()
                durations[k] = v

        self.durations = durations

    def convert_phoneme(self, sentence: str) -> tuple:
        """Returns first phone from the sentence"""

        if sentence[0] == " ":
            return ("_", 1)
        elif self.equivs:
            # s is a special case, needs to peek next
            if sentence[0] == "s":
                try:
                    if sentence[1] in "aeiouyAEIOUY&,":
                        return ("s", 1)
                except IndexError:
                    pass

            for equiv in self.equivs.items():
                if re.match(re.escape(equiv[0]), sentence):
                    # print("match:", equiv[0], phoneme, "=", equiv[1])
                    return (equiv[1], len(equiv[0]))

        return (equiv[0], 1)


    def get_duration(self, phoneme: str, factor: float) -> float:
        if self.durations and phoneme in self.durations:
            return int(float(self.durations[phoneme]) * factor)
        else:
            return 100 * factor

    def convert_sentence(self, input_str: str) -> Sentence:
        key = 125
        octaves = 1.2
        decl = 0

        factor = 1 / 1.0
        freq = key

        freq_t = key * sqrt(2 ** octaves)
        freq_m = key
        freq_b = key / sqrt(2 ** octaves)

        sentence = Sentence()
        sentence.phones.append(Phone(" ", "_", 150 * factor, [[50, key]]))

        ignored = ["@", "\n", ",", "'", "^", ";"]

        for word in input_str.split():
            print(word)
            labeled = re.match(r"\[([TMBHSLUD])\]", word)
            if labeled:
                word = re.sub(r"\[[TMBHSLUD]\]", "", word)
                label = labeled.group(1)
                if label == "T":
                    freq = freq_t
                elif label == "M":
                    freq = freq_m
                elif label == "B":
                    freq = freq_b
                elif label == "H":
                    print(freq)
                    freq = sqrt(freq * freq_t)
                    print(freq)
                elif label == "S":
                    freq = freq
                elif label == "L":
                    freq = sqrt(freq * freq_b)
                elif label == "U":
                    freq = sqrt(freq * sqrt(freq * freq_t))
                elif label == "D":
                    freq = sqrt(freq * sqrt(freq * freq_b))

            sampa = self.text_to_sampa(word)
            sampa = sampa.replace("'", "")
            print(";; ", sampa)

            while sampa:
                if sampa[0] not in ignored:
                    converted = self.convert_phoneme(sampa)
                    phone_sampa = sampa[:converted[1]]
                    sampa = sampa[converted[1]:]

                    duration = self.get_duration(converted[0], factor)
                    phone = Phone(
                        phone_sampa = phone_sampa,
                        phone_mbrola = converted[0],
                        duration = duration,
                        pitch_changes = [[50, freq + decl]],
                        stress = False# percentage, Hz
                    )

                    sentence.phones.append(phone)
                else:
                    sampa = sampa[1:]
                decl -= 1

        sentence.phones.append(Phone(" ", "_", 150 * factor, [[50, freq]]))
        return sentence

    def text_to_sampa(self, sentence: str) -> str:
        espeak_str = "espeak-ng -v pt-br '{}' -x -q".format(sentence)

        p = subprocess.Popen(espeak_str, stdout=subprocess.PIPE, shell=True)
        (output, err) = p.communicate()
        p_status = p.wait()
        output = output.decode("utf-8")
        output = output.replace("\n", "").strip()

        return output

if __name__ == "__main__":
    converter = Converter()

    for line in sys.stdin:
        print(";;", line)
        print(converter.convert_sentence(line))
\end{lstlisting}

\begin{lstlisting}[caption=Script para editor gráfico, label=editorjs, language=Python]
function getRandomInt(max) {
  return Math.floor(Math.random() * Math.floor(max));
}

var self = this;
const requestURL = "http://127.0.0.1:5000/api/espeak";
let XHR = new XMLHttpRequest();
let FD  = new FormData();

let textBox = document.querySelector("#textBox");

let audio = document.querySelector("#audio");
let audioSource = document.querySelector("#audioSource");

let postDiv = document.querySelector("#post");
postDiv.style.display = "none";

let genBtn = document.querySelector("#generateBtn");

let drawArea = document.querySelector("#drawArea");

genBtn.addEventListener("click", function( event ) {
  while (drawArea.firstChild) {
    drawArea.removeChild(drawArea.firstChild);
  }

  console.log(textBox.value);
  const requestURL = "http://127.0.0.1:5000/api/gen_audio";

  let XHR = new XMLHttpRequest();
  let FD  = new FormData();
  FD.append("text", textBox.value);

  XHR.open("POST", requestURL);
  XHR.send(FD);

  XHR.onreadystatechange = function() {
    if (XHR.readyState === XMLHttpRequest.DONE) {
      if (XHR.status === 200) {
        const results = JSON.parse(XHR.responseText);
        console.log(results);
        console.log(results.token);

        let curLen = 0;
        let height = 250;

        for (let phone of results.sentence) {
          console.log(phone);
          let txtX = curLen + (phone.duration / 2);

          var svgns = "http://www.w3.org/2000/svg";
          var rect = document.createElementNS(svgns, 'rect');
          rect.setAttributeNS(null, 'x', curLen + phone.duration);
          rect.setAttributeNS(null, 'y', 0);
          rect.setAttributeNS(null, 'height', height);
          rect.setAttributeNS(null, 'width', '2');
          rect.setAttributeNS(null, 'fill', '#CCCCCC');
          drawArea.appendChild(rect);

          var circ = document.createElementNS(svgns, 'circle');
          var cx = curLen + ((phone.pitch_changes[0][0] / 100.0) * phone.duration);
          circ.setAttributeNS(null, 'cx', cx);
          circ.setAttributeNS(null, 'cy', height - phone.pitch_changes[0][1]);
          circ.setAttributeNS(null, 'r', 5);
          circ.setAttributeNS(null, 'fill', '#777777');
          circ.setAttributeNS(null, 'class', "dot");
          drawArea.appendChild(circ);

          var text = document.createElementNS(svgns, 'text');
          var tx = curLen + (phone.duration / 2);
          text.setAttributeNS(null, 'x', cx);
          text.setAttributeNS(null, 'y', 20);
          text.setAttributeNS(null, 'text-anchor', "middle");
          text.setAttributeNS(null, 'style', "font: 14px sans-serif");
          text.setAttributeNS(null, 'fill', '#777777');
          text.textContent = phone.phone_mbrola;
          drawArea.appendChild(text);

          curLen += phone.duration;
        }
        console.log(curLen);
        drawArea.setAttribute("width", `${curLen}px`);

        audioSource.src = `http://127.0.0.1:5000/api/get_audio/${results.token}`;
        audio.load();
        postDiv.style.display = "block";
      }
    }
  };

}, false);
\end{lstlisting}

\begin{lstlisting}[caption=Exemplo de resposta para \emph{endpoint} do eSpeakNG, label=espeakpost, language=Python]
[
  {
    "duration": 150,
    "phone_mbrola": "_",
    "phone_sampa": " ",
    "pitch_changes": [[50, 150]
    ]
  },
  {
    "duration": 80,
    "phone_mbrola": "n",
    "phone_sampa": "n",
    "pitch_changes": [[50, 150]]
  },
  {
    "duration": 110,
    "phone_mbrola": "u",
    "phone_sampa": "U",
    "pitch_changes": [[50, 150]]
  },
  ...
  {
    "duration": 150,
    "phone_mbrola": "_",
    "phone_sampa": " ",
    "pitch_changes": [[50, 150]]
  }
]
\end{lstlisting}

\begin{lstlisting}[caption=Exemplo de resposta para \emph{endpoint} do MBROLA, label=mbrolaget, language=Python]
{
  "mp3_file": "07acc5c4ba924294.mp3"
}
\end{lstlisting}

\begin{lstlisting}[caption=Gerador de $ f_0 $ a partir do modelo INTSINT, label=intsintpy, language=Python]
from math import sqrt

def intsint(labels: list, key: float = 150.0, range: float = 1.0):
    if labels[0] not in ["T", "M", "B"]:
        raise ValueError("First item in list can't be relative")

    freq_t = key * sqrt(2 ** range)
    freq_m = key
    freq_b = key / sqrt(2 ** range)

    freqs = []

    for idx, l in enumerate(labels):
        if l == "T":
            freqs.append(freq_t)
        elif l == "M":
            freqs.append(freq_m)
        elif l == "B":
            freqs.append(freq_b)
        elif l == "H":
            freqs.append(sqrt(freqs[idx - 1] * freq_t))
        elif l == "S":
            freqs.append(freqs[idx - 1])
        elif l == "L":
            freqs.append(sqrt(freqs[idx - 1] * freq_b))
        elif l == "U":
            freqs.append(sqrt(freqs[idx - 1] * sqrt(freqs[idx - 1] * freq_t)))
        elif l == "D":
            freqs.append(sqrt(freqs[idx - 1] * sqrt(freqs[idx - 1] * freq_b)))

    return freqs
\end{lstlisting}
