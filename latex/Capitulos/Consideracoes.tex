% Considerações finais
% As considerações finais formam a parte final (fechamento) do texto, sendo dito de forma resumida (1) o que foi desenvolvido no presente trabalho e quais os resultados do mesmo, (2) o que se pôde concluir após o desenvolvimento bem como as principais contribuições do trabalho, e (3) perspectivas para o desenvolvimento de trabalhos futuros. O texto referente às considerações finais do autor deve salientar a extensão e os resultados da contribuição do trabalho e os argumentos utilizados estar baseados em dados comprovados e fundamentados nos resultados e na discussão do texto, contendo deduções lógicas correspondentes aos objetivos do trabalho, propostos inicialmente.

\chapter{Considerações finais}
Neste trabalho foi apresentada a definição de um sistema TTS, bem como a descrição
de seus componentes principais e possíveis implementações para cada módulo.
Também foi realizada uma busca dos sistemas TTS \emph{open-source} e comerciais
existentes. Com o objetivo de melhorar a geração de prosódia, foi feita uma
revisão de prosódia na linguística, descrevendo os principais sistemas de
anotação usados na fonologia entoacional, principalmente aplicados ao português
brasileiro. Investigamos como a prosódia funciona nos sistemas encontrados e
identificamos que, enquanto a geração de prosódia suprassegmental em trabalhos
recentes já produz resultados satisfatórios, ainda há desafios quando à síntese
de fala expressiva relacionada à prosódia afetiva e aumentativa.

\section{Trabalhos futuros}
Foi identificado que ainda há um significante desafio para a geração de prosódia
afetiva automática a partir de um texto não anotado. Possíveis soluções para
avanço nesta área incluem:
\begin{itemize}
\item Expansão do sistema desenvolvido neste trabalho para adicionar suporte a
  mais modelos notação entoacional.
\item Avaliação estatística da qualidade de falas geradas com a aplicação desenvolvida comparada a outros sistemas \emph{open-source} e comerciais
\item Adicionar suporte a SSML e EmotionML ao sistema desenvolvido, gerando
  curvas F0 a partir de linguagens de marcação
\item Uso técnicas de técnicas de \emph{Natural Language Understanding} para gerar notação EmotionML automaticamente.
\item Desenvolvimento de corpus anotados com prosódia para o português brasileiro.
\end{itemize}

% Usar técnicas de Natural Language Understanding para gerar prosódia utilizando notação desenvolvida neste trabalho
% corpus anotados para o português
% https://corplinguistics.wordpress.com/2012/03/08/prosodically-annotated-corpora/