% Considerações finais
% As considerações finais formam a parte final (fechamento) do texto, sendo dito de forma resumida (1) o que foi desenvolvido no presente trabalho e quais os resultados do mesmo, (2) o que se pôde concluir após o desenvolvimento bem como as principais contribuições do trabalho, e (3) perspectivas para o desenvolvimento de trabalhos futuros. O texto referente às considerações finais do autor deve salientar a extensão e os resultados da contribuição do trabalho e os argumentos utilizados estar baseados em dados comprovados e fundamentados nos resultados e na discussão do texto, contendo deduções lógicas correspondentes aos objetivos do trabalho, propostos inicialmente.

\chapter{Considerações finais}
Neste trabalho apresentamos a definição de TTS,
fizemos uma busca dos sistemas open-source existentes
investigamos como a prosódia funciona nos sistemas encontrados
especialmente para o português brasileiro
investigamos como melhorar
passos para o futuro
implementação básica de uma provável melhoria

\section{Trabalhos futuros}
Usar técnicas de Natural Language Understanding para gerar notação EmotionML
automaticamente.
Desenvolvimento de corpus anotados com prosódia para o
português brasileiro.
% Usar técnicas de Natural Language Understanding para gerar prosódia utilizando notação desenvolvida neste trabalho
% corpus anotados para o português
% https://corplinguistics.wordpress.com/2012/03/08/prosodically-annotated-corpora/