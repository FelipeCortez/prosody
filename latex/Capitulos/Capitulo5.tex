% Capítulo 5 - Resultados
% devem ser apresentados, de forma objetiva, precisa e clara, tanto os resultados positivos quanto os negativos que foram obtidos com o desenvolvimento do trabalho, sendo feita uma discussão que consiste na avaliação circunstanciada, na qual se estabelecem relações, deduções e generalizações.

\chapter{Resultados}

Com o levantamento dos sistemas TTS desenvolvidos para o português brasileiro, é
possível observar que há carência de suporte à síntese expressiva. Seguindo o modelo de
\cite{taylor2009}, os sistemas estudados concentram-se na geração de prosódia
suprassegmental, mas não há suporte algum à geração de prosódia afetiva.

Linguagens de marcação como SSML e EmotionML já foram integradas a
\emph{frameworks} para TTS e sistemas comerciais, mas ainda não há integração
entre essas linguagens e síntese de voz para o português brasileiro.

Estudando os modelos de anotação entoacional, percebe-se que ainda não há uma
solução mais adequada para analisar o português brasileiro de forma a 

Com o desenvolvimento da aplicação \emph{web} deste trabalho, dá-se um passo
inicial para a integração de prosódia afetiva a sistemas TTS para o português brasileiro.
% necessidade de corpus prosódico aberto para português brasileiro: corpus
% utilizado por trabalho da Microsoft não é disponibilizado \cite{hmmmicrosoft}