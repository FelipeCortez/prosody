% Capítulo 5 - Resultados e trabalhos futuros
% devem ser apresentados, de forma objetiva, precisa e clara, tanto os resultados positivos quanto os negativos que foram obtidos com o desenvolvimento do trabalho, sendo feita uma discussão que consiste na avaliação circunstanciada, na qual se estabelecem relações, deduções e generalizações.

\chapter{Capítulo 5}

\section{Resultados}
\subsection{Metodologia}
% Pesquisa feita online, arquivos de áudio, soluções comerciais, soluções open-source
\subsection{MOS}
% Comparar MOS (Mean Opinion Score) com outras soluções

\section{Trabalhos futuros}
% Usar técnicas de Natural Language Understanding para gerar prosódia utilizando notação desenvolvida neste trabalho

% \cite{siri} é um programa.
% 
% Na tese de Doutorado de Paquete \cite{PaquetePhD}, discute-se sobre algoritmos de busca local estocásticos aplicados a problemas de Otimização Combinatória considerando múltiplos objetivos. Por sua vez, o trabalho de \cite{KnowlesBoundedLebesgue}, publicado nos anais do IEEE CEC de 2003, mostra uma técnica de arquivamento também empregada no desenvolvimento de algoritmos evolucionários multi-objetivo, trabalho esse posteriormente estendido para um capítulo de livro dos mesmos autores \cite{KnowlesBoundedPareto}. Por fim, no relatório técnico de \citeonline{Jaszkiewicz}, fala-se sobre um algoritmo genético híbrido para problemas multi-critério, enquanto no artigo de jornal de Lopez \textit{et al.} \cite{LopezPaqueteStu} trata-se do \textit{trade-off} entre algoritmos genéticos e metodologias de busca local, também aplicados no contexto multi-critério e relacionado de alguma forma ao trabalho de Jaszkiewicz (\citeyear{Jaszkiewicz}).
% 
% Outros exemplos relacionados encontram-se em \cite{Silberschatz} (livro), \cite{DB2XML} (referência da Web) e \cite{Angelo} (dissertação de Mestrado).