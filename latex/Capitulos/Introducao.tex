% Introdução

% A introdução é a parte inicial do texto e que possibilita uma visão geral de todo o trabalho, devendo constar a delimitação do assunto tratado, objetivos da pesquisa, motivação para o desenvolvimento da mesma e outros elementos necessários para situar o tema do trabalho.

\chapter{Introdução}
% - Este trabalho é uma tentativa de melhorar o módulo de prosódia em sistemas TTS
% para o português brasileiro
% - Investigação da prosódia no ptbr
% - Investigação de sistemas TTS para o português
% - Motivação: artificialidade devida à prosódia ``neutra''

Interfaces humano-computador que utilizam a voz, denominadas \emph{voice user interfaces}, antigamente vistas apenas na ficção científica, hoje são uma realidade e estão disponíveis em \emph{smartphones} e ambientes \emph{desktop}. De acordo com \textcite{tts-book, martinjurafsky}, há uma grande área de aplicação para essas interfaces, destacando-se a acessibilidade, permitindo que deficientes visuais possam ouvir texto sem a necessidade de gravação prévia de seu conteúdo. Além disso, com o aumento da popularidade de sistemas embarcados, é importante investigar novas formas de interação humano-máquina, e a síntese de fala, juntamente com o reconhecimento, permitem comunicação de duas vias com esses sistemas. Sistemas \emph{text-to-speech} (doravante designados pela sigla TTS) também podem servir para pessoas que perderam a habilidade de falar, como o físico Stephen Hawking, que desde 1986 utilizou um sintetizador de voz para se comunicar, e o crítico de cinema Roger Ebert, que após perder a mandíbula passou a falar através de um sistema TTS, mais tarde usando uma solução personalizada que sintetizava uma aproximação de sua própria voz baseada em múltiplas gravações passadas. 

Os serviços mais populares e robustos que temos atualmente são implementações proprietárias de grandes empresas, como Siri \parencite{siri}, Cortana \parencite{cortana} e Alexa \parencite{alexa}. Apesar da praticidade e ganho de acessibilidade providos por essas interfaces, os serviços disponíveis sintetizam voz com resultados perceptivelmente artificiais, principalmente para a língua portuguesa, se compararmos com os mesmos serviços para o inglês. Uma das causas da artificialidade é a prosódia empregada, isto é, o ritmo, entonação e acento. Mesmo com um sistema personalizado, Roger Ebert se queixava da falta de expressividade do algoritmo.

\subsection{Objetivos}
Neste trabalho propõe-se investigar as causas da artificialidade de prosódia em sistemas TTS disponíveis, estudando os modelos de prosódia, algoritmos e métodos para síntese de fala para diversas línguas com foco no português, e implementar um ou mais modelos de geração de prosódia promissores para o português brasileiro baseados na estrutura sintática do texto de entrada identificada por técnicas de processamento de linguagem natural. Ao final da implementação, os resultados serão avaliados qualitativamente através de questionários, compa\-rando-os ao estado da arte e disponibilizando o sistema publicamente.

\section{Organização do trabalho}

- Capítulo 2 - contextualização - breve história:
ferramentas para TTS em português, tipos de prosódia, dificuldades, modelos de
análise e síntese

- Capítulo 3 - referencial:
revisão da literatura, incluindo sistemas TTS existentes, estudos de prosódia

- Capítulo 4 - metodologia:
justificativa para abordagem escolhida, implementação, ferramentas, arquitetura
do software desenvolvido

- Capítulo 5 - resultados:
testes MOS, metodologia do teste, trabalhos futuros, mapeamento automático

% Nesta seção deve ser apresentado como está organizado o trabalho, sendo descrito, portanto, do que trata cada capítulo.