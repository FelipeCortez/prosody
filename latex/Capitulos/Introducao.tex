% Introdução

\chapter{Introdução}
- Este trabalho é uma tentativa de melhorar o módulo de prosódia em sistemas TTS
para o português brasileiro
- Investigação da prosódia no ptbr
- Investigação de sistemas TTS para o português
- Motivação: artificialidade devida à prosódia ``neutra''

% A introdução é a parte inicial do texto e que possibilita uma visão geral de todo o trabalho, devendo constar a delimitação do assunto tratado, objetivos da pesquisa, motivação para o desenvolvimento da mesma e outros elementos necessários para situar o tema do trabalho.

\section{Organização do trabalho}

- Capítulo 2 - contextualização - breve história:
ferramentas para TTS em português, tipos de prosódia, dificuldades, modelos de
análise e síntese

- Capítulo 3 - referencial:
revisão da literatura, incluindo sistemas TTS existentes, estudos de prosódia

- Capítulo 4 - metodologia:
justificativa para abordagem escolhida, implementação, ferramentas, arquitetura
do software desenvolvido

- Capítulo 5 - resultados:
testes MOS, metodologia do teste, trabalhos futuros, mapeamento automático

% Nesta seção deve ser apresentado como está organizado o trabalho, sendo descrito, portanto, do que trata cada capítulo.