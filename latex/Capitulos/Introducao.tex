% Introdução

% A introdução é a parte inicial do texto e que possibilita uma visão geral de todo o trabalho, devendo constar a delimitação do assunto tratado, objetivos da pesquisa, motivação para o desenvolvimento da mesma e outros elementos necessários para situar o tema do trabalho.

\abrv[TTS -- \emph{text-to-speech}]

\chapter{Introdução}
% - Este trabalho é uma tentativa de melhorar o módulo de prosódia em sistemas TTS
% para o português brasileiro
% - Investigação da prosódia no ptbr
% - Investigação de sistemas TTS para o português
% - Motivação: artificialidade devida à prosódia ``neutra''

Interfaces humano-computador que utilizam a voz, denominadas \emph{voice user
  interfaces}, antigamente vistas apenas na ficção científica, hoje são uma
realidade e estão disponíveis em \emph{smartphones} e ambientes \emph{desktop}.
De acordo com \cite{tts-book, martinjurafsky}, há uma grande área de aplicação
para essas interfaces, como a acessibilidade, permitindo que
deficientes visuais possam ouvir texto sem a necessidade de gravação prévia de
seu conteúdo, ferramenta de ensino de linguagens e auxílio à pesquisa na linguística. Além disso, com o aumento da popularidade de sistemas embarcados,
é importante investigar novas formas de interação humano-máquina, e a síntese de
fala, juntamente com o reconhecimento, permitem comunicação de duas vias com
esses sistemas.

% Sistemas \emph{text-to-speech} (doravante TTS) também podem
% auxiliar pessoas que perderam a habilidade de falar, como o físico Stephen
% Hawking, que desde 1986 utilizou um sintetizador de voz para se comunicar, e o
% crítico de cinema Roger Ebert, que após perder a mandíbula passou a falar
% através de um sistema TTS, mais tarde usando uma solução personalizada que
% sintetizava uma aproximação de sua própria voz baseada em múltiplas gravações
% passadas. % referência! 

Os serviços mais populares e robustos que temos atualmente são implementações proprietárias de grandes empresas, como Siri \cite{siri}, Cortana \cite{cortana} e Alexa \cite{alexa}. Apesar da praticidade e ganho de acessibilidade providos por essas interfaces, os serviços disponíveis sintetizam voz com resultados perceptivelmente artificiais, principalmente para a língua portuguesa. Uma das causas da artificialidade é a prosódia empregada, isto é, o ritmo, entonação e acento.

\section{Objetivos}
Este trabalho tem como objetivo geral propor melhorias para o módulo de prosódia afetiva para sistemas TTS com suporte ao português brasileiro através de uma revisão da área da fonologia e estudo de estado da arte de sistemas TTS, com atenção especial a como a prosódia é abordada.

\section{Organização do trabalho}
% Nesta seção deve ser apresentado como está organizado o trabalho, sendo descrito, portanto, do que trata cada capítulo.

% - Capítulo 2 - contextualização
No capítulo 2, é feita uma breve explanação dos conceitos abordados neste
trabalho, com foco em TTS e prosódia.

% ferramentas para TTS em português, tipos de prosódia, dificuldades, modelos de
% análise e síntese

% - Capítulo 3 - referencial:
% revisão da literatura, incluindo sistemas TTS existentes, estudos de prosódia
No capítulo 3, é feita uma revisão da literatura, fazendo um levantamento dos
sistemas \emph{text-to-speech} existentes tanto para o inglês quanto para o
português brasileiro e como a prosódia é modelada em cada um deles. Mostramos
como trabalhos recentes abordam síntese de fala. Também são descritos os
trabalhos existentes em análise de prosódia em um contexto não necessariamente
computacional.

% - Capítulo 4 - metodologia:
No capítulo 4, é descrito o sistema desenvolvido, justificando a abordagem com
base na revisão da literatura. Explica-se a implementação do software,
descrevendo sua arquitetura, as linguagens de programações utilizadas e o funcionamento.

% - Capítulo 5 - resultados:
% $testes MOS, metodologia do teste, trabalhos futuros, mapeamento automático
No capítulo 5,
% explicitamos os resultados, descrevendo a metodologia empregada
% na avaliação qualitativa das amostras de áudio geradas pelo \emph{software}
% desenvolvido comparadas ao estado da arte.
apresentamos os resultados obtidos com a implementação.